\documentclass[twoside,letterpaper,10pt]{article}

\usepackage{mystyle}

\title{The Kolmogorov Theorem}
\author{Travis Westura}
\date{\today}

\begin{document}

\maketitle

\section{Introduction}
\label{sec:introduction}

\begin{abstract}
  This paper gives a proof of the Kolmogorov Theorem on the conservation of
  invariant tori.
  We follow the approach given by Hubbard and Ilyashenko in .
  % TODO write citation.
\end{abstract}

\begin{thm}[The Kolmogorov Theorem]
  \label{thm:KAM}
  Let $\rho, \gamma > 0$ be given, and let $h(\bp{q}, \bp{p}) = h_0(\bp{p}) +
  h_1(\bp{q}, \bp{p})$ be a Hamiltonian, with $h_0, h_1 \in \mathcal{A}_{\rho}$
  and $\norm{h}_{\rho} \leq 1$.
  Suppose the Taylor polynomial of $h_0$ is
  \begin{equation*}
    h_0(\bp{p}) = a + \omega\bp{p} + \frac{1}{2} \bp{p} \cdot C \bp{p} +
    o(|\bp{}|^2),
  \end{equation*}
  with $\omega \in \Omega_{\gamma}$ and $C$ is symmetric and invertible.
  Then for any $\rho_* \leq \rho$, there exists $\epsilon > 0$, which depends on
  $C$ and $\gamma$, but not on the remainder term in $o(|\bp{p}|^2)$, such that
  if $\norm{h_1}_{\rho} \leq \epsilon$, there exists a symplectic mapping $\Phi
  : A_{\rho_*} \to A_{\rho}$ such that if we set $(\bp{q}, \bp{p} = \Phi(\bp{Q},
  \bp{P})$ and $H = h \circ \Phi$, we have
  \begin{equation*}
    H(\bp{Q}, \bp{P}) = A + \omega \bp{P} + R(\bp{Q}, \bp{P}),
  \end{equation*}
  with $R(\bp{Q}, \bp{P}) \in O(|\bp{P}|^2)$.
\end{thm}

\begin{proof}
  The proof is left as an exercise for the reader.
\end{proof}

\end{document}

%%% Local Variables:
%%% mode: latex
%%% TeX-master: t
%%% End:
