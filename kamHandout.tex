\documentclass[twoside,letterpaper,10pt]{article}

\usepackage{mystyle}

\expandafter\def\expandafter\normalsize\expandafter{%
    \normalsize
    \setlength\abovedisplayskip{2pt}
    \setlength\belowdisplayskip{2pt}
    \setlength\abovedisplayshortskip{2pt}
    \setlength\belowdisplayshortskip{2pt}
}

\title{Kolmogorov's Theorem}
\author{Travis Westura}
\date{\today}

\newcommand{\KAM}{ Let $\rho, \gamma > 0$ be given, and let
  $h(\bp{q}, \bp{p}) = h_0(\bp{p}) + h_1(\bp{q}, \bp{p})$ be a Hamiltonian, with
  $h_0, h_1 \in \mathcal{A}_{\rho}$ and $\norm{h}_{\rho} \leq 1$.
  Suppose the Taylor polynomial of $h_0$ is
  \begin{equation*}
    h_0(\bp{p}) = a + \omega\bp{p} + \frac{1}{2} \bp{p} \cdot C \bp{p} +
    o(|\bp{p}|^2),
  \end{equation*}
  with $\omega \in \Omega_{\gamma}$ and $C$ is symmetric and invertible.
  Then for any $\rho_* \leq \rho$, there exists $\epsilon > 0$, which depends on
  $C$ and $\gamma$, but not on the remainder term in $o(|\bp{p}|^2)$, such that
  if $\norm{h_1}_{\rho} \leq \epsilon$, there exists a symplectic mapping $\Phi
  : A_{\rho_*} \to A_{\rho}$ such that if we set $(\bp{q}, \bp{p}) = \Phi(\bp{Q},
  \bp{P})$ and $H = h \circ \Phi$, we have
  \begin{equation*}
    H(\bp{Q}, \bp{P}) = A + \omega \bp{P} + R(\bp{Q}, \bp{P}),
  \end{equation*}
  with $R(\bp{Q}, \bp{P}) \in O(|\bp{P}|^2)$.
}

\newcommand{\dionumber}{
  A number $\theta$ is \emph{Diophantine of exponent d} if there exists a
    constant $\gamma > 0$ such that for all coprime integers $p$ and $q$ we have
    \begin{equation*}
      \left| \theta - \frac{p}{q} \right| > \frac{\gamma}{|q|^d}.
    \end{equation*}
}

\newcommand{\diovector}{
  Let $\omega = (\omega_1, \ldots, \omega_n)$.
    We say $\omega$ is Diophantine if there exists $\gamma > 0$ such that for
    all vectors with integer coefficients $(k_1, \ldots, k_n)$, we have
    \begin{equation*}
      |k_1 \omega_1 + \cdots + k_n \omega_n| \geq \frac{\gamma}{(k_1^2 + \cdots
        + k_n^2)^{\frac{n}{2}}}.
    \end{equation*}
    Let $\Omega_{\gamma}^n$ be the subset of such $\omega \in \R^n$.
}

\newcommand{\domains}{
  \begin{align*}
    B_{\rho} &= \{\bp{p} \in \C : |\bp{p}| \leq \rho\},\\
    C_{\rho} &= \{\bp{q} \in \C^n / \Z^n : | \Imag(\bp{q}) | \leq \rho\},\\
    A_{\rho} &= C_{\rho} \times B_{\rho} = \{(\bp{q}, \bp{p}) \in \C^n / \Z^n
               \times \C^n : |\bp{p}| \leq \rho, \, |\Imag(\bp{q})| \leq \rho\}.
  \end{align*}
}


\newcommand{\sgrad}{\nabla_{\sigma}}

\begin{document}

\maketitle

Our goal is to understand Kolmogorov's theorem:
\begin{thm}[Kolmogorov's Theorem]
  \KAM
\end{thm}
In particular the motion
\begin{align*}
  \bp{Q}(t) &= \bp{Q}(0) + t\omega_0,\\
  \bp{P}(t) &= 0,
\end{align*}
is a solution of the Hamiltonian equation that is conjugate to the linear flow
with direction $\omega_0$, so that the invariant torus $\bp{p} = 0$ is preserved
by the perturbation.

To read and then understand this theorem, we require some definitions and
notation.

If $(X, \sigma)$ is a symplectic manifold, then any function $H$ on $X$ has a
symplectic gradient $\sgrad H$ defined as the unique vector field such that for
any vector field $\xi$, it holds that $\sigma(\xi, \sgrad H) = d H(\xi)$.
The Hamiltonian differential equation is $\dot{x} = (\sgrad H)(x)$.

$\sgrad f$ has a flow $\phi^t_f$ such that $f \circ \phi^t_f = f$ and
$(\phi^t_f)^* \sigma = \sigma$.

\begin{defn}[Poisson Bracket]
The \emph{Poisson Bracket} of $X$ is defined by
  \begin{equation*}
    \{f, g\} = \sigma(\sgrad g, \sgrad f) = df(\sgrad g) = - dg(\sgrad f).
  \end{equation*}
\end{defn}
Functions $f$ and $g$ \emph{commute} if $\{f, g\} = 0$.
This condition implies that their flows commute:
\begin{equation*}
  \phi_f(s) \circ \phi_g(t) = \phi_g(t) \circ \phi_f(s).
\end{equation*}
The symplectic gradient of the Poisson bracket is the Lie bracket:
\begin{equation*}
  \sgrad\{f, g\} = [\sgrad f, \sgrad g].
\end{equation*}
The Lie bracket $[X, Y]$ is the derivative of $Y$ in the ``direction'' of $X$.

\begin{exmp}[Hamiltonian Equations of Motion]
  Take $X = \R^{2n}$ with coordinates
  $(\bp{q}, \bp{p}) = (q_1, \ldots, q_n, p_1, \ldots, p_n)$ and
  $\sigma = \sum_i \dif p_i \wedge \dif q_i$.
  Then the Hamiltonian differential equation becomes the \emph{Hamiltonian
    Equations of Motion}:
  \begin{align*}
    \dot{q}_i &= \pd{H}{p_i},\\
    \dot{p}_i &= -\pd{H}{q_i}.
  \end{align*}
  We can integrate these equations.
  The solution with initial conditions $(\bp{q}_0, \bp{p}_0)$ is
  \begin{align*}
    \bp{q}(t) &= \bp{q}_0 + t \pd{H}{\bp{p}}(\bp{p}_0) =: \bp{q}_0 + t
                \omega(\bp{p}_0),\\
    \bp{p}(t) &= \bp{p}_0.
  \end{align*}
  Each coordinate $p_1, \ldots p_n$ is conserved, and the motion is a linear
  motion on the torus $\mathbb{T}^n \times \{\bp{p}_0\}$.
  
  In this case the Poisson bracket is given by
  \begin{equation*}
    \{f, g\} = \sum_{i = 1}^n \left( \pd{f}{q_i} \pd{g}{p_i} - \pd{f}{p_i}
      \pd{g}{q_i} \right).
  \end{equation*}
\end{exmp}
The Poisson bracket allows us to write Taylor Series as
\begin{equation*}
  f \circ \phi^t_g = f + t \{f, g\} + \frac{t^2}{2} \{ \{f, g\}, g\} +
  \frac{t^3}{3!} \{ \{ \{f, g\}, g\}, g\} + \cdots.
\end{equation*}

\begin{defn}[Diophantine Number of Exponent $d$]
  \dionumber{}
\end{defn}

\begin{defn}[Diophantine Vector]
  \diovector
\end{defn}

If an analytic function $f$ is bounded on an open set $U$ and if $V$ is
relatively compact in $U$, then we can bound the second derivatives of $f$ on
$V$ in terms of $\sup_U |f|$.
The important domains are
\domains{}
Denote by $\mathcal{B}_\rho$,
$\mathcal{C}_\rho$, and $\mathcal{A}_\rho$ the corresponding Banach algebras of
functions continuous on these compact sets and analytic on the interiors, with
sup-norm $\norm{f}_{\rho}$.
\begin{defn}[Banach Algebra]
  Let $k$ be $\R$ or $\C$.
  A \emph{normed algebra} over $k$ is an algebra $\mathcal{A}$ over $k$ with a
  sub-multiplicative norm $\norm{.}$, that is, for all $x, y \in \mathcal{A}$,
  we have $\norm{xy} \leq \norm{x} \norm{y}$,
  If $\mathcal{A}$ is a Banach space, then it is called a \emph{Banach algebra}.
\end{defn}
Elements of $\mathcal{B}_{\rho}$ can be expanded as Power series.
Elements of $\mathcal{C}_{\rho}$ can be expanded as Fourier series
$f(\bp{z})~=~\sum_{\bp{k} \in \Z^{2n}} f_{\bp{k}} \e^{2 \pi \i \, \bp{k} \cdot
  \bp{z}}$.

\end{document}

%%% Local Variables:
%%% mode: latex
%%% TeX-master: t
%%% End:
